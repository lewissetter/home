\documentclass{article}
\usepackage{amsmath}
\usepackage[left=1in, right=1in, top=1in]{geometry}

\title{Homework 1, EENG 515, Fall 2018}
\author{Lewis Setter}

\begin{document}

\maketitle

\begin{enumerate}
\item
Done

\item 
Let $R = S \cap T$ where $S$ and $T$ are both convex sets. Take two points $s$ and $t$
within the intersection. We know the points satisfy both
\[\lambda s + (1-\lambda)t \in S\]
and
\[\lambda s + (1-\lambda)t \in T\]
when
\[0 \leq \lambda \leq 1\]
By definition, every point in $R$ is in both $S$ and $T$. Therefore every point in
the intersection also satisfies
\[\lambda s + (1-\lambda)t \in R\]
when
\[0 \leq \lambda \leq 1\]

\item
Let $A = [0, 1)$ and let $B = (1, 2]$. If $a = 0$, $b = 2$, and $\lambda = 0.5$, we 
have
\[\lambda a + (1-\lambda)b = 1\]
when
\[0 \leq \lambda \leq 1\]
Hence, 
\[\lambda a + (1-\lambda) b \notin A \cup B\]
when
\[0 \leq \lambda \leq 1\]

\item
Let us assume that there exists at least one rational number $x$ such that $x + \sqrt{2}$ 
is a rational number. This implies that there are a set of integer values $a$, $b$, $c$, 
and $d$ that satisfy the following.
\begin{equation}\label{root2IsRational}
\begin{split}
\frac{a}{b} + \sqrt{2} = \frac{c}{d} \\
\sqrt{2} = \frac{cb-ad}{bd}
\end{split}
\end{equation}
Given the products, quotients, and differences of rational numbers are also rational
numbers, the last line of (\ref{root2IsRational}) indicates that $\sqrt{2}$ is a rational
number. Hence, our initial assumption that there is a least one rational number that
yields a rational result to the sum $x + \sqrt{2}$ is incorrect.

\item
Let $m^{2}$ be even. Now suppose $m$ is odd. This means there is an integer $a$ such 
that $m = 2a + 1$. If this is the case, then the following holds.
\begin{equation}
\begin{split}
m^{2} = (2a + 1)^{2} \\
= 4a^{2} + 4a + 1 \\
= 2(2a^{2} + 2a) + 1
\end{split}
\end{equation}
which indicates the $m^{2}$ is odd since $2a^{2} + 2a$ is an integer. This implies that 
$m$ must be even.
\end{enumerate}
\end{document}
