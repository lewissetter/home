\documentclass{article}
\usepackage[margin=1in]{geometry}
\usepackage{listings}
\usepackage{color}
 
\definecolor{codegreen}{rgb}{0,0.6,0}
\definecolor{codegray}{rgb}{0.5,0.5,0.5}
\definecolor{codepurple}{rgb}{0.58,0,0.82}
\definecolor{backcolour}{rgb}{0.95,0.95,0.92}
 
\lstdefinestyle{mystyle}{
    backgroundcolor=\color{backcolour},   
    commentstyle=\color{codegreen},
    keywordstyle=\color{magenta},
    stringstyle=\color{codepurple},
    basicstyle=\footnotesize,
    breakatwhitespace=false,         
    breaklines=true,                 
    captionpos=b,                    
    keepspaces=true,                 
    showspaces=false,                
    showstringspaces=false,
    showtabs=false,                  
    tabsize=2
}

\lstset{style=mystyle}

\title{Answers to the Intro to Open CV Assignment}
\author{Lewis Setter}

\begin{document}

\maketitle

\begin{enumerate}
\item
OpenCV stores images useing the Mat class.

\item
\textit{image.shape} returns a tuple containing the number of rows, columns, and 
channels in \textit{image}.

\item
\textit{fx} and \textit{fy} in \textit{c2v.resize} are scale factors for resize the
horizontal and vertical axes, respectively.

\item
In the specified case, the resulting rotation matrix would describe a rotation by the
same amount but about a point toward the upper left corner of the image it affects.

\item

\item
Functions in python are written like this:
\begin{lstlisting}[language=Python]
    def functionName (arg 1, arg2, arg3, areYouGettingTheIdea):
        # make sure you indent once you finish the function delcaration

        #
        # code within the function goes here!
        #

        return optionalVal
\end{lstlisting}
\item
It depends on where you decide to look!
A function in Python always execute using the same set of arguments definied in the
the fucntion definition. However, a programmer does not have to pass in every argument
to the function if they have taken advantage of default values. For instance, consider
the function definition below:
\begin{lstlisting}[language=Python]
    def exampleFunction(SEED, lab = "awesome!"):
        ...
\end{lstlisting}
When the program calls this function, it is required to pass in at least one argument
so that "SEED" has a value. They do not need to pass in two arguments because "lab"
has a default value of "awesome!". However, the function is free to take two parameters.
In this case, "lab" will be assigned whatever the second argument in the calling code is.

NOTE: If default values are used for some arguments in a function definition, those
arguments must be listed consecutively as the last arguments in the definition.

\item
Here are two methods that can be used with list functions:
\\
sort(): returns a sorted version of the input list\\
any(): checks if any of the list items are logically equivalent to true\\

\end{enumerate}
\end{document}

imamge.shape:
https://docs.opencv.org/3.0-beta/doc/py_tutorials/py_core/py_basic_ops/py_basic_ops.html
