\documentclass{article}
\usepackage[margin=1in]{geometry}
\usepackage{listings}
\usepackage{color}
 
\definecolor{codegreen}{rgb}{0,0.6,0}
\definecolor{codegray}{rgb}{0.5,0.5,0.5}
\definecolor{codepurple}{rgb}{0.58,0,0.82}
\definecolor{backcolour}{rgb}{0.95,0.95,0.92}
 
\lstdefinestyle{mystyle}{
    backgroundcolor=\color{backcolour},   
    commentstyle=\color{codegreen},
    keywordstyle=\color{magenta},
    stringstyle=\color{codepurple},
    basicstyle=\footnotesize,
    breakatwhitespace=false,         
    breaklines=true,                 
    captionpos=b,                    
    keepspaces=true,                 
    showspaces=false,                
    showstringspaces=false,
    showtabs=false,                  
    tabsize=2
}

\lstset{style=mystyle}

\title{Answers to the Intro to Pi Assignment}
\author{Lewis Setter}

\begin{document}

\maketitle

\begin{enumerate}
\item
An integrated development environment (IDE) is a software package which allows a user
to write programs with the help of tools provided by the IDE. These tools vary from
one IDE to another, but typical examples are linters, debuggers, and autocompleters.

\item
The try/except structure in Python is the primary mean for exception handling. Code 
which may throw an exception is placed within the scope of the "try" block. Likewise,
code which is able to elegantly handle this potential exception is placed within the 
scope of the "except" block. When the "try" code throws an exception, the "except"
code is run with the exception as a parameter.

\item
Functions in python are written like this:
\begin{lstlisting}[language=Python]
    def functionName (arg 1, arg2, arg3, areYouGettingTheIdea):
        # make sure you indent once you finish the function delcaration

        #
        # code within the function goes here!
        #

        return optionalVal
\end{lstlisting}
\item
It depends on where you decide to look!
A function in Python always execute using the same set of arguments definied in the
the fucntion definition. However, a programmer does not have to pass in every argument
to the function if they have taken advantage of default values. For instance, consider
the function definition below:
\begin{lstlisting}[language=Python]
    def exampleFunction(SEED, lab = "awesome!"):
        ...
\end{lstlisting}
When the program calls this function, it is required to pass in at least one argument
so that "SEED" has a value. They do not need to pass in two arguments because "lab"
has a default value of "awesome!". However, the function is free to take two parameters.
In this case, "lab" will be assigned whatever the second argument in the calling code is.

NOTE: If default values are used for some arguments in a function definition, those
arguments must be listed consecutively as the last arguments in the definition.

\item
Here are two methods that can be used with list functions:
\\
sort(): returns a sorted version of the input list\\
any(): checks if any of the list items are logically equivalent to true\\

\end{enumerate}
\end{document}
