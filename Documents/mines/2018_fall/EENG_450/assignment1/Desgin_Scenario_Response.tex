\documentclass[11pt]{article}
\usepackage[margin=1in]{geometry}

\title{Design Scenario Response}
\author{Lewis Setter}

\begin{document}

\maketitle

\section{High Level Project Plan}\label{project_plan}
The first steps I would take involve getting a grasp on the scope of the problem and
my resources. I would reach out to the previous team members, their advisor, and others
who worked on the project initially to determine the challenges I would face, the 
resources they found useful, and broader intentions of the project.

Once I had a clear Idea of the problem, I would reach out to people in control of the
resources that I am interested in for my continuation of the project. I would get in
touch with member of the ADAPT center, for instance, and figure out the ways in which
they could be of service.

After defining the problem at hand and the resources available, I would begin to consider
solutions. In doing so I would continue to work with my resources at the ADAPT center,
the Mines faculty, and so forth. I would take a diverse set of considerations into
account including technical, ethical, financial, and customer driven schools of thought.

With a general idea of the solution in mind, I could begin the design process.
Throughout this cycle, I would continue to work with and continuously evaluate my
available resources. This is especially true of manufacturers, who by this point would
likely be one of my greatest concerns. The ability of manufacturers to produce my
design would limit my possible solution space further and further as time on the 
project expired.

After developing a working model, I would continue to tweak and improve as long as
time permitted. I would place an order to my chosen manufacturers and wait for the
product to arrive. While waiting, I imagine that I would be working on perfecting my
documentation, creating a series of presentations for any relevant parties, and making
sure manufacturing was going smoothly. Then, when the final product came in, I would
be able to successfully meet all the requirements of the project.

\section{Potential Issues and Problem Solving Framework}
With very limited exception, long term projects do not typically turn out as originally
expected. In terms of the project in question, the following list of problems would be
some of my foremost concerns:
\begin{itemize}
\item Parties related to the original project being difficult to work with
\item A slower than expected prototyping cycle
\item High manufacturing times
\item Frequent demands by the customer to change the direction of the development
\item The customers understanding of their own needs versus what they think they need
\end{itemize}
These are obviously just a few of the potential problems that could seriously affect
the outcome of the project. As such, I have decided to lay out of phases discussed in
Section~\ref{project_plan} as specifically as possible without making any commitments
that rely on details not provided in project description. I would continue to hold
this philosophy throughout the project in hopes that it would provided the flexibility
necessary to elegantly deal with any issues, anticipated or otherwise.

\end{document}
